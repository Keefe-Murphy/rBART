\documentclass{article}

\usepackage{fullpage}
\usepackage{graphics}
\usepackage{amsmath}
\usepackage{indentfirst}
\usepackage{setspace}
\usepackage[]{algorithm2e}

\DeclareMathOperator{\var}{var}
\DeclareMathOperator{\cov}{cov}
\newcommand{\bm}[1]{\mbox{\boldmath $ #1 $}}
\newcommand{\EE}{\mathrm{I\!E\!}}
\newcommand{\RR}{\mathrm{I\!R\!}}
\parindent=0in

\doublespacing
\pagenumbering{arabic}

\begin{document}

\title{BART MCMC updates}
\author{Andrew C. Parnell}
\maketitle


\section*{Overall algorithm details}

The equations referred to below are displayed later in this document.

\RestyleAlgo{boxruled}
\begin{algorithm}[H]
 \KwData{Target variables $y$ (length $n$; standardised), feature matrix $X$ ($n$ rows and $p$ columns}
 \KwResult{Posterior list of trees $T$, values of $\tau$, fitted values $\hat{y}$}
 \textbf{Initialisation\;}
 	Hyper-parameter values of $\alpha$, $\beta$, $\tau_\mu$, $\nu$, $\lambda$\;
	Number of trees $M$\;
	Number of iterations $N$\;
	Initial value $\tau = 1$\;
	Set trees $T_j$; $j=1, \ldots, M$ to stumps\;
	Set values of $\mu$ to 0\;
\For{iterations $i$ from 1 to $N$}{
	\For{trees $j$ from 1 to $M$}{
		Compute partial residuals from $y$ minus predictions of all trees except tree $j$\;
		Grow a new tree $T_j^{new}$ based on grow/prune/change/swap\;
		Set $l_{new}$ = log full conditional of new tree $T^{new}_j$ based on Equation 1 plus Equation 4\;
		Set $l_{old}$ = log full conditional of old tree $T_j$ based on same equations\;
		Set $a = \exp(l_{new} - l_{old} )$\;
		Generate $u \sim U(0, 1)$\;
		\If{$a > u$}{
			Set $T_j = T_j^{new}$\;
		}
		Simulate $\mu$ values using Equation 3\;
	}
	
	Get predictions $\hat{y}$ from all trees\;
	Update $\tau$ using Equation 4\;
}
\caption{BART Markov chain Monte Carlo}
\end{algorithm}


\section{Updating trees}

Suppose there are $n$ observations in a terminal node and suppose that the (partial) residuals in this terminal node are denoted $R_1, \ldots, R_n$. The prior distribution for these residuals is:
$$ R_1, \ldots, R_n | \mu, \tau \sim N(\mu, \tau^{-1})$$
Furthermore the prior on $\mu$ is:
$$\mu \sim N(0, \tau_{\mu}^{-1})$$

Using $\pi$ to denote a probability distribution, we want to find:
\begin{align*}
\pi(R_1, \ldots, R_n| \tau) &= \int \pi(R_1, \ldots, R_n| \mu, \tau) \pi(\mu) \partial \mu \\
& \propto \int \prod_{i=1}^n \tau^{1/2} e^{-\frac{\tau}{2} (R_i - \mu)^2} \tau_{\mu}^{1/2} e^{-\frac{\tau_\mu}{2} \mu^2 } \partial \mu \\
&= \int \tau^{n/2} e^{-\frac{\tau}{2} \sum (R_i - \mu)^2} \tau_{\mu}^{1/2} e^{-\frac{\tau_\mu}{2} \mu^2 } \partial \mu \\
&= \int \tau^{n/2} \tau_{\mu}^{1/2} e^{-\frac{1}{2} \left[ \tau \left\{ \sum R_i^2 + n \mu^2 - 2 \mu n \bar{R} \right\} + \tau_\mu \mu^2 \right]} \partial \mu \\
&= \tau^{n/2} \tau_{\mu}^{1/2} e^{-\frac{1}{2} \left[ \tau \sum R_i^2 \right]} \int e^{-\frac{1}{2} Q} \partial \mu 
\end{align*}
where 
\begin{align*}
Q &= \tau n \mu^2 - 2\tau n \mu \bar{R} + \tau_\mu \mu^2 \\
&= (\tau_\mu + n \tau) \mu^2 - 2\tau n \mu \bar{R} \\
&= (\tau_\mu + n \tau) \left[ \mu^2 - \frac{2 \tau n \mu \bar{R}}{ \tau_\mu + n \tau } \right]\\
&= (\tau_\mu + n \tau) \left[ \left( \mu - \frac{2 \tau n \bar{R}}{ \tau_\mu + n \tau } \right)^2 - \left( \frac{ \tau n \bar{R} }{ \tau_\mu + n \tau} \right)^2 \right]\\
&= (\tau_\mu + n \tau) \left( \mu - \frac{2 \tau n \bar{R}}{ \tau_\mu + n \tau } \right)^2 - \frac{(n \tau \bar{R})^2}{ \tau_\mu + n \tau}\\
\end{align*}
so therefore:
\begin{align*}
 \int e^{-\frac{1}{2} Q} \partial \mu &= \int \exp \left[  - \frac{\tau_\mu + n \tau}{2} \left( \mu - \frac{2 \tau n \bar{R}}{ \tau_\mu + n \tau } \right)^2 + \frac{(n \tau \bar{R})^2}{2( \tau_\mu + n \tau)} \right] \partial \mu \\
\propto &  \exp \left[ \frac{1}{2} \frac{ (\tau n \bar{R})^2 }{ \tau_\mu + n \tau } \right] (\tau_\mu + n \tau)^{-1/2}
\end{align*}

And finally:
\begin{align*}
\pi(R_1, \ldots, R_n | \tau) \propto & (\tau_\mu + n \tau)^{-1/2} \tau^{n/2} \tau_\mu^{1/2} \exp \left[ \frac{1}{2} \frac{ (\tau n \bar{R})^2 }{ \tau_\mu + n \tau } \right] \exp \left[ -\frac{\tau}{2} \sum R_i^2 \right] \\
&= \tau^{n/2} \left( \frac{\tau_\mu}{\tau_\mu + n \tau} \right)^{1/2} \exp \left[ -\frac{\tau}{2} \left\{ \sum R_i^2 - \frac{ \tau (n\bar{R})^2 }{ \tau_\mu + n \tau } \right\} \right]
\end{align*}

\subsection*{Including multiple terminal nodes}

When we put back in terminal nodes we write $R_{ji}$ where $j$ is the terminal node and $i$ is still the observation, so in terminal node $j$ we have partial residuals $R_{j1}, \ldots, R_{jn_j}$. When we have $j=1,\ldots,b$ terminal nodes the full conditional distribution is then:

\begin{align*}
\prod_{j=1}^b \pi(R_{j1}, \ldots, R_{jn_j} | \tau) &\propto \prod_{j=1}^b \left\{ \tau^{n_j/2} \left( \frac{\tau_\mu}{\tau_\mu + n_j \tau} \right)^{1/2} \exp \left[ -\frac{\tau}{2} \left\{ \sum_{i=1}^{n_j} R_{ji}^2 - \frac{ \tau (n_j \bar{R}_j)^2 }{ \tau_\mu + n_j \tau } \right\} \right] \right\}
\end{align*}
which on the log scale gives:
\begin{align*}
\sum_{j=1}^b \left\{ \frac{n_j}{2} \log(\tau) + \frac{1}{2} \log \left( \frac{\tau_\mu}{\tau_\mu + n_j \tau} \right) - \frac{\tau}{2} \left[ \sum_{i=1}^{n_j} R_{ji}^2 - \frac{ \tau (n_j \bar{R}_j)^2 }{ \tau_\mu + n_j \tau } \right] \right\}
\end{align*}

This can be simplified further to give:
\begin{align}
\frac{n}{2} \log(\tau) + \frac{1}{2} \sum_{j=1}^b \log \left( \frac{\tau_\mu}{\tau_\mu + n_j \tau} \right) - \frac{\tau}{2} \sum_{j=1}^b 
\sum_{i=1}^{n_j} R_{ji}^2 + \frac{\tau^2}{2} \sum_{j=1}^b \frac{ S_j^2 }{ \tau_\mu + n_j \tau }
\end{align}

where $S_j = \sum_{i=1}^{n_j} R_{ji}$


\subsection*{Updating $\mu$}

The full conditional for $\mu_{j}$ (the terminal node parameters for node $j$) is similar to the above but without the integration:

\begin{align*}
\pi(\mu_{j}|\ldots)  &\propto \prod_{i=1}^{n_j}  \tau^{1/2} e^{-\frac{\tau}{2} (R_{ji} - \mu_{j})^2} \tau_{\mu}^{1/2} e^{-\frac{\tau_\mu}{2} \mu_{j}^2 }\\
&\propto e^{-\frac{\tau}{2} \sum_{i=1}^{n_j} (R_{ji} - \mu_{j})^2} e^{-\frac{\tau_\mu}{2} \mu_{j}^2 } \\ 
&\propto e^{-\frac{\tau}{2} \left[ n_j \mu_{j}^2 - 2 \mu_j  \sum_{i=1}^{n_j} R_{ji} \right] - \frac{\tau_\mu}{2} \mu_j^2}\\
&\propto e^{-\frac{Q}{2}}
\end{align*}
Now:
\begin{align*}
Q &= n_j \tau \mu_j^2 - 2\mu_j \tau S_j + \tau_\mu \mu_j^2\\
&= (n_j \tau + \tau_\mu) \mu_j^2 - 2 \tau \mu_j S_j \\
&= (n_j \tau + \tau_\mu) \left[ \mu_j^2 - \frac{2 \tau \mu_j S_j}{n_j \tau + \tau_\mu} \right] \\
&\propto (n_j \tau + \tau_\mu) \left[ \mu_j - \frac{\tau \mu_j S_j}{n_j \tau + \tau_\mu} \right]^2
\end{align*}
so therefore:
\begin{align}
\mu_j| \ldots \sim N \left( \frac{\tau S_j}{n_j \tau + \tau_\mu} , (n_j \tau + \tau_\mu)^{-1} \right)
\end{align}

\subsection*{Update for $\tau$}

I am using the shape/rate parameterisation of the gamma with prior $\tau \sim Ga(\nu/2, \nu \lambda / 2)$. Letting $\mu_i$ be the prediction of the $i$th observation we get:
\begin{align*}
\pi(\tau | \ldots) &\propto \prod_{i=1}^n \tau^{1/2} e^{-\frac{\tau}{2} (y_i - \mu_i)^2} \tau^{\nu/2 - 1} e^{-\tau \nu \lambda / 2}
\end{align*}
Letting $S = \sum_{i=1}^n (y_i - \mu_i)^2$ we get:
\begin{align*}
\pi(\tau | \ldots) &\propto \tau^{n/2} e^{-\frac{\tau}{2} S} \tau^{\nu/2 - 1} e^{-\tau \nu \lambda / 2}\\
&= \tau^{(n + \nu)/2 - 1} e^{-\frac{\tau}{2} (S + \nu \lambda)}
\end{align*}
so
\begin{align}
\tau | \ldots \sim Ga \left( \frac{n + \nu}{2}, \frac{S + \nu \lambda}{2} \right)
\end{align}

\subsection*{Tree prior}

The tree prior used by BARTMachine says that the probability of a node being non-terminal is:
\begin{align*}
P(\mbox{node is non-terminal}) &= \alpha (1 + d)^{-\beta}
\end{align*}
So the probability of a node being terminal is 1 minus this. A stump just has probability 1 - $\alpha$. For Bart Machine $\alpha = 0.95$ and $\beta = 2$

Thus for a tree with $k$ non-terminal nodes and $b$ terminal nodes we have:
\begin{align*}
P &= \prod_{i=1}^b \left[ 1 - \alpha (1 + d^t_i)^{-\beta} \right] \prod_{i=1}^k \left[ \alpha (1 + d^{nt}_i)^{-\beta} \right] \\
\end{align*}
where $d^t_i$ is the depth of the $i$th terminal node and $d^{nt}_i$ is the depth of the $i$th non-terminal node.
On the log scale this gives:
\begin{align}
\log P &= \sum_{i=1}^b \left[ \log \left(1 - \alpha(1 + d^t_i)^{-\beta} \right) \right] + \sum_{i=1}^k \left[ \log(\alpha) - \beta \log(1 + d^{nt}_i) \right]
\end{align}

\section*{2-class classification version}

A 2-class classification version can be created by using the latent probit representation of the binomial distribution. We assume the response variable $y_i$ takes values 0 or 1, and depends on a latent variable $z_i$ upon which the BART model is applied:
$$y_i = \left\{ \begin{array}{ll} 1 & \mbox{ if } z_i > 0 \\ 0 & \mbox{ otherwise} \end{array} \right.$$
now $z_i \sim N \left(\sum_{j=1}^m g_j(x_i), 1\right)$ where $g_j$ are the predicted values from the individual trees $j$. Note that this model implies that $\tau = 1$ and is not updated.

The update for $z_i$ at the end of each set of tree updates is:
\begin{eqnarray}
z_i | y_i = 1 &\sim& max \left[ N\left(\sum_{j=1}^m g_j(x_i), 1\right), 0 \right] \\
z_i | y_i = 0 &\sim& min \left[ N\left(\sum_{j=1}^m g_j(x_i), 1\right), 0 \right]
\end{eqnarray}

Otherwise all updates (trees and means) are the same. However, finding the predicted probabilities at the end of the algorithm requires $\phi^{-1}(\sum_{j=1}^m g_j(x_i))$ where $\phi$ is the normal cdf function. These probabilities can be then be used to create misclassification tables, ROC curves, etc.

\section*{Mean and variance changing}

An alternative, slightly more flexible, model can be created by giving each terminal node it's own precision value. Thus for tree $j$ we have:
$$ R_{1j}, \ldots, R_{jn_j} | \mu_j, \tau_j \sim N(\mu_j, \tau_j^{-1})$$
The maths is much simplified by setting the prior on $\mu_j$ as:
$$\mu_j \sim N(0, (a \tau_j)^{-1}).$$
Chipman et al (1998) set $a = 1/3$ for a single tree setting, so perhaps $a = 1/(3m)$ might be more appropriate for a multi-tree version with $m$ trees.\\

We now have a prior for each terminal node:
$$\tau_j \sim Ga(\nu/2, \nu \lambda / 2)$$
Chipman et al (1998) suggest making $\nu$ and $\lambda$ functions of tree complexity but we don't do that here.\\

We don't need the subscripts $j$ for an individual tree so the shortcut to what we require is:
\begin{align*}
\pi(R_1, \ldots, R_n| \ldots ) &= \int \int \pi(R_1, \ldots, R_n| \mu, \tau) \pi(\mu) \pi(\tau) \partial \mu \; \partial \tau \\
&\propto \int \int \left[ \prod_{i=1}^n \tau^{1/2} \exp \left(-\frac{\tau}{2} (R_i - \mu)^2 \right) \right]\tau^{1/2} \exp \left(-\frac{a \tau}{2} \mu^2 \right) \tau^{\nu/2 - 1} \exp \left( - \frac{\tau \nu \lambda} {2} \right) \partial \mu \; \partial \tau \\
&\propto \int \int \tau^{n/2} \exp \left(-\frac{\tau}{2} \sum (R_i - \mu)^2 \right) \tau^{1/2} \exp \left(-\frac{a \tau}{2} \mu^2 \right) \tau^{\nu/2 - 1} \exp \left( - \frac{\tau \nu \lambda} {2} \right) \partial \mu \; \partial \tau 
\end{align*}
Note that $\sum (R_i - \mu)^2$ can be re-written as $SS_{R} + n(\mu - \bar{R})^2$ where $SS_R = \sum (R_i - \bar{R})^2$.\\

We now get
\begin{align*}
\pi(R_1, \ldots, R_n| \ldots ) &\propto \int \int  \tau^{(n+\nu - 1)/2} \exp \left(-\frac{\tau}{2} \left\{ SS_{R} + n(\mu - \bar{R})^2  + a \mu^2 + \nu \lambda \right\} \right) \partial \mu \; \partial \tau
\end{align*}

Simplifying the exponent to separate out the terms with $\mu$ gives:
$$n(\mu - \bar{R})^2 + a \mu^2 = (a+n)\left[ \mu - \frac{n\bar{R}}{n + a} \right]^2 + \frac{a n\bar{R}^2}{n + a}$$

So we can perform the integrand with respect to $\mu$ first:
\begin{align*}
\pi(R_1, \ldots, R_n| \ldots ) &\propto  \int  \tau^{(n+\nu - 2)/2} \exp \left(-\frac{\tau}{2} \left\{ SS_{R} + \nu \lambda + \frac{an\bar{R}^2}{n + a} \right\} \right) \int \tau^{1/2} \exp \left(-\frac{\tau (a+n)}{2} \left[ \mu - \frac{n\bar{R}}{n + a} \right]^2  \right) \partial \mu \; \partial \tau
\end{align*}
The integrand wrt $\mu$ is proportional to $(a + n)^{-1/2}$ and also provides the full conditional for $\mu$ (putting back in the subscripts):
$$\mu_j |\ldots \sim N \left( \frac{n_j\bar{R}_j}{n_j + a} , \left[ \tau_j (a+n_j) \right]^{-1} \right)$$

Next we have: 
\begin{align*}
\pi(R_1, \ldots, R_n| \ldots ) &\propto  \int  \frac{\tau^{(n+\nu - 2)/2}}{(a+n)^{1/2}} \exp \left(-\frac{\tau}{2} \left\{ SS_{R} + \nu \lambda + \frac{an\bar{R}^2}{n + a} \right\} \right)  \partial \tau
\end{align*}

This also provides the complete conditional for (re-inserting subscripts) $\tau_j$:
$$\tau_j |\ldots \sim Ga \left( \frac{n_j + \nu}{2} , \frac{1}{2} \left[ SS_{R_j} + \nu \lambda + \frac{an_j\bar{R}_j^2}{n_j+a} \right] \right)$$

Finally we have (again with subscripts):
\begin{align*}
\pi(R_{1j}, \ldots, R_{jn_j}| \ldots ) &\propto (a+n_j)^{-1/2} \; \Gamma \left( \frac{n_j + \nu}{2} \right) \left[SS_{R_j} + \nu \lambda + \frac{an_j\bar{R}_j^2}{n_j + a} \right]^{-\left( \frac{n_j + \nu}{2} \right)}
\end{align*}

With multiple terminal nodes for a tree we have:
\begin{align*}
\prod_{j=1}^b \pi(R_{1j}, \ldots, R_{jn_j}| \ldots ) &= \prod_{j=1}^b (a+n_j)^{-1/2} \; \Gamma \left( \frac{n_j + \nu}{2} \right) \left[SS_{R_j} + \nu \lambda - \frac{an_j\bar{R}_j^2}{n_j + a} \right]^{-(\frac{n_j + \nu}{2})} \\
&= \left[ \prod_{j=1}^b (a+n_j)^{-1/2} \right] \left[ \prod_{j=1}^b \Gamma \left( \frac{n_j + \nu}{2} \right) \right] \left\{ \prod_{j=1}^b \left[SS_{R_j} + \nu \lambda + \frac{an_j\bar{R}_j^2}{n_j + a} \right]^{-\left( \frac{n_j + \nu}{2} \right)} \right\}
\end{align*}

which on the log scale is:
$$\log \prod_{j=1}^b \pi_j = -\frac{1}{2} \sum_{j=1}^b \log(a + n_j) + \sum_{j=1}^b \log \Gamma \left( \frac{n_j + \nu}{2} \right) - \sum_{j=1}^b \left( \frac{n_j + \nu}{2} \log \left[SS_{R_j} + \nu \lambda + \frac{an_j\bar{R}_j^2}{n_j + a} \right] \right) $$

\end{document}


